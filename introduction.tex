\chapter{Introduction}

Communication is a fundamental part of human society, enabling the transfer of information between people.
Textual language in particular is found throughout our daily lives;
when asking a question, we type it into a search engine and usually read an answer from either Wikipedia or discussions between other people;
to find out what is happening in the world we read newspapers or listen to a newsreader;
for entertainment we read stories, listen to podcasts, or watch TV shows (virtually all of which contain dialogue);
and for personal communcation we write electronic messages in a variety of formats.
One of the keys to all of this communication is the use of structure to convey meaning.
Part of this structure is visible--by convention you are looking at this page from left to right and top to bottom, treating each connected component of ink (or light) as a character, combining a sequence of characters into words, and so on to progressively larger levels of structure.
The focus of this dissertation, syntax, is the mostly invisible structure that exists somewhere between characters and sentences.
People are able to identify syntactic structure by drawing on knowledge from a range of sources, and in the process they discard a vast array of alternative interpretations for a given utterance.
In the past century, artificial symbolic processing by machines has advanced dramatically, to the point where computers can start to engage in textual communication, both understanding what people have written and writing back.
Approaching human-level communication will require systems to resolve syntactic ambiguities in text, either explicitly with an interpretable syntactic structure or implicitly with some other intermediate representation.

Over the past two decades there has been rapid development in systems for the syntactic parsing task, where the input is a single sentence and the output is a structure that encodes syntactic relationships between words in the sentence.
This development has largely been driven by development of new statistical methods for constructing models of language from resources manually labeled with syntactic structures by linguists.
In addition to variations in statistical and engineering approaches, research has explored various syntactic representations.
These representations are based on different linguistic theories, which sometimes encode the same phenomena in different ways and sometimes differ on what phenomena should be disambiguated within syntax.

In this thesis, we present new algorithms that extend the capabilities of various systems related to syntax.
The first area of research we consider is how we evaluate our systems to understand their strengths and weaknesses.
We propose a search algorithm that finds the set of errors present in a parse.
Using the algorithm, we compare a range of systems, a range of text domains, and two languages.
The second aspect of syntax we consider is conversion between different linguistic formalisms.
Our system uses a bottom-up approach that more flexible and more effective than previous work.
Finally, we explore how to extend parsing algorithms beyond tree structured output.
We introduce a new algorithm that can efficiently find the max or sum over all possible structures for a sentence, scored with an edge-factored model.
We define a new representation that is compatible with the algorithm while representing virtually all forms of linguistic structure in the Penn Treebank.

\section{Syntax}
 - representation of meaning
 - constituency v dependency
 - CFG v mildly context sensitive
 - ambiguity (meaningful and spurious)

\section{Error Analysis}

Constituency parser performance is primarily interpreted through a single
metric, F-score on \wsj section 23, that conveys no linguistic information
regarding the remaining errors.  We classify errors within a set of
linguistically meaningful types using tree transformations that repair groups
of errors together.  We use this analysis to answer a range of questions about
parser behaviour, including what linguistic constructions are difficult for
state-of-the-art parsers, what types of errors are being resolved by rerankers,
and what types are introduced when parsing out-of-domain text.


Parsing has been a major area of research within computational linguistics for
decades, and constituent parser F-scores on \wsj section 23 have exceeded
$90\%$ \parencite{Petrov-Klein:2007}, and $92\%$ when using self-training and
reranking \parencite{McClosky-Charniak-Johnson:2006,Charniak-Johnson:2005}. 
While these results give a useful measure of overall performance, they provide
no information about the nature, or relative importance, of the remaining
errors.

Broad investigations of parser errors beyond the \parseval metric
\parencite{Black-etal:1991} have either focused on specific parsers, \eg
\parencite{Collins:2003}, or have involved conversion to dependencies
\parencite{Carroll-etal:1998,King:2003}.  In all of these cases, the analysis has
not taken into consideration how a set of errors can have a common cause, \eg a
single mis-attachment can create multiple node errors.

We propose a new method of error classification using tree transformations.
Errors in the parse tree are repaired using subtree movement, node
creation, and node deletion.  Each step in the process is then associated with
a linguistically meaningful error type, based on factors such as the node that is
moved, its siblings, and parents.  

Using our method we analyse the output of thirteen constituency parsers on
newswire.  Some of the frequent error types that we identify are widely recognised
as challenging, such as prepositional phrase (PP) attachment.  However, other
significant types have not received as much attention, such as clause
attachment and modifier attachment.

Our method also enables us to investigate where reranking and self-training
improve parsing.  Previously, these developments were analysed only in terms of
their impact on F-score.  Similarly, the challenge of out-of-domain parsing has
only been expressed in terms of this single objective.  We are able to
decompose the drop in performance and show that a disproportionate number of
the extra errors are due to coordination and clause attachment.

This work presents a comprehensive investigation of parser behaviour in terms
of linguistically meaningful errors.  By applying our method to multiple
parsers and domains we are able to answer questions about parser behaviour that
were previously only approachable through approximate measures, such as counts
of node errors.  We show which errors have been reduced over the past fifteen
years of parsing research; where rerankers are making their gains and where
they are not exploiting the full potential of k-best lists; and what types of
errors arise when moving out-of-domain.  We have released our
system\footnote{http://code.google.com/p/berkeley-parser-analyser/} to enable
future work to apply our methodology.

\section{Conversion}

\section{Parsing}

General treebank analyses are graph structured, but parsers are typically restricted to tree structures for efficiency and modeling reasons.
We propose a new representation and algorithm for a restricted class of graph structures that are flexible enough to cover almost all treebank structures, yet are restricted enough to admit efficient learning and inference.
In particular, we consider directed, acyclic, one-endpoint-crossing graph structures, which cover most dislocation, shared argumentation, and similar tree-violating linguistic phenomena.
We describe how to convert phrase structure parses, including traces, to our new representation, in a reversible manner.
Our dynamic program uniquely decomposes structures, is sound and complete with respect to a subset of the class of one-endpoint-crossing graphs, and covers $97.7\%$ of the Penn English treebank.
We also implement a proof-of-concept parser that recovers a range of null elements and trace types.

Many syntactic representations use graphs and/or discontinuous structures, such as traces in Government and Binding theory, and f-structure in Lexical Functional Grammar \parencite{gb,Bresnan:1982}.
Sentences in the Penn Treebank \parencite{ptb} have a core projective tree structure, and trace edges that represent control structures, wh-movement and more.
However, most parsers and the standard evaluation metric ignore these edges and all null elements.
By leaving out parts of the structure, they fail to provide all predicates to downstream tasks such as question answering.
While there has been work on capturing some parts of this extra structure, it has generally either been through post-processing on trees
\parencite{Johnson:2002,Jijkoun:2003,Campbell:2004,Levy:2004,Gabbard:2006},
or has only captured a limited set of phenomena via grammar augmentation
\parencite{collins:1997,dienes-dubey:2003,schmid:2006,cai-chiang-goldberg:2011}.
In both cases phenomena such as shared argumentation are completely ignored.
Similarly, most work on the Abstract Meaning Representation \parencite{amr}, has removed edges to turn all structures into trees.

\begin{figure}
  \centering
  \scalebox{0.65}{
  \begin{tikzpicture}
[ every node/.style={
    node distance=2ex,
    inner sep=0pt
  },
  structural/.style={
    ->,
    >=stealth',
    thin,
  },
  trace/.style={
    ->,
    >=stealth',
    thin,
    dashed
  },
]

% Split head form
  \node (sntR0) at (0, 1.8) {\strut ROOT};

  \node (snt00) [right=1em of sntR0.east] {\strut NP\textsubscript{SBJ}};
  \node (sw0) [below=-0.3ex of snt00] {\strut Sam};
%%%  \draw (sw0.north) -- (snt00.south);

  \node (snt10) [right=1em of snt00.east] {\strut S};
  \node (snt11) [right=1ex of snt10.east] {\strut VP};
  \draw (snt10.east) -- (snt11.west);
  \node (sw1a) at ($(snt10.west)!0.5!(snt11.east)$) {\strut};
  \node (sw1) [below=-0.3ex of sw1a] {\strut told};
%%%  \draw (sw1.north) -- (sw1a.south);

  \node (snt20) [right=1em of snt11.east] {\strut NP};
  \node (sw2) [below=-0.3ex of snt20] {\strut Alex};
%%%  \draw (sw2.north) -- (snt20.south);

  \node (snt30) [right=1.5em of snt20.east] {\strut -};
  \node (sw3) [below=-0.3ex of snt30] {\strut to};
%%%  \draw (sw3.north) -- (snt30.south);

  \node (snt50) [right=1em of snt30.east] {\strut S};
  \node (snt51) [right=1ex of snt50.east] {\strut VP};
  \draw (snt50.east) -- (snt51.west);
  \node (snt52) [right=1ex of snt51.east] {\strut VP};
  \draw (snt51.east) -- (snt52.west);
  \node (sw5a) at ($(snt50.west)!0.5!(snt52.east)$) {\strut};
  \node (sw5) [below=-0.3ex of sw5a] {\strut run};
%%%  \draw (sw5.north) -- (sw5a.south);

  \draw [structural,out=60,in=120] (snt00.north) to (snt10.north west);
  \draw [structural,out=120,in=60] (snt10.north) to (sntR0.north);
  \draw [structural,out=120,in=60] (snt20.north) to (snt11.north east);
  \draw [trace,out=30,in=150] (snt20.north) to node[midway,above=1pt] {\scriptsize NP\textsubscript{SBJ}} (snt50.north west);
  \draw [structural,out=120,in=60] (snt50.north) to (snt11.north);
  \draw [structural,out=60,in=120] (snt30.north) to (snt51.north);

% Standard form
  \node (w0) [above=8ex of sw0] {\strut Sam};
  \node (w1) [right=3ex of w0.east] {\strut told};
  \node (w2) [right=3ex of w1.east] {\strut Alex};
  \node (w3) [right=3ex of w2.east] {\strut};
  \node (w4) [right=3ex of w3.east] {\strut to};
  \node (w5) [right=3ex of w4.east] {\strut run};

  \node (nt00) [above=of w0.north] {\strut};
  \node (nt10) [above=of w1.north] {\strut};
  \node (nt20) [above=of w2.north] {\strut};
  \node (nt30) [above=of w3.north] {\strut *\textsubscript{1}};
  \node (nt40) [above=of w4.north] {\strut};
  \node (nt50) [above=of w5.north] {\strut VP};

  \node (nt01) [above=of nt00.north] {\strut};
  \node (nt11) [above=of nt10.north] {\strut};
  \node (nt21) [above=of nt20.north] {\strut};
  \node (nt31) [above=of nt30.north] {\strut NP\textsubscript{SBJ}};
  \path (nt40.north) -- node[above=2ex] (nt41-51) {\strut VP} (nt50.north);

  \node (nt02) [above=of nt01.north] {\strut};
  \node (nt12) [above=of nt11.north] {\strut};
  \node (nt22) [above=of nt21.north] {\strut NP\textsubscript{1}};
  \path (nt31.north) -- node[above=2ex] (nt32-52) {\strut S} (nt41-51.north);

  \node (nt03) [above=of nt02.north] {\strut NP\textsubscript{SBJ}};
  \path (nt12.north) -- node[above=2ex] (nt13-53) {\strut VP} (nt32-52.north);

  \path (nt03.north) -- node[above=2ex] (nt04-54) {\strut S} (nt13-53.north);

  \draw (w0.north) -- (nt03.south);
  \draw (w1.north) -- (nt12.north) -- (nt13-53.south) -- (nt32-52.north);
  \draw (w2.north) -- (nt22.south);
  \draw (w4.north) -- (nt40.north) -- (nt41-51.south) -- (nt50.north);
  \draw (w5.north) -- (nt50.south);

  \draw (nt03.north) -- (nt04-54.south) -- (nt13-53.north);
  \draw (nt22.north) -- (nt13-53.south);
  \draw (nt30.north) -- (nt31.south);
  \draw (nt31.north) -- (nt32-52.south) -- (nt41-51.north);

  \node (figB) [left=4ex of sntR0] {\strut \large (b)};
  \node (figA) at (figB |- nt31) {\strut \large (a)};
\end{tikzpicture}

  }
  \vspace{-5mm}
  \caption{\label{fig:repr}
    Parse representation: (a) constituency (b) ours.
  }
\end{figure}

We propose a new parse representation and a new algorithm that can efficiently consider almost all syntactic phenomena.
Our representation is an extension of TAG-based tree representations \parencite{cck,Shen:2007}, modified to represent graphs and designed to maximize coverage under a new class of graphs.
Our algorithm extends a non-projective tree parsing algorithm \parencite{ec} to graph structures, with improvements to avoid derivational ambiguity.

Our representation, shown in Figure~\ref{fig:repr}b, consists of complex tags composed of non-terminals, and edges indicating attachment.
In this form, traces can create problematic structures such as directed cycles, but we show how careful choice of head rules can minimize such issues.

Our algorithm runs in time $O(n^4)$ under a first-order model.
We also introduce extensions that ensure parses contain a directed projective tree of non-trace edges.
We implemented a proof-of-concept parser with a basic first-order model, scoring $88.3$ on trees in section 22, and recovering a range of trace types.

Together, our representation and algorithm form an inference method that can cover $97.7\%$ of sentences, far above the coverage of projective tree algorithms ($46.8\%$).

\section{Contributions}

