\chapter{Introduction}


\section{Graph Parsing}

\subsection{Old abstract}
General treebank analyses are graph structured, but parsers are typically restricted to tree structures for efficiency and modeling reasons.
We propose a new representation and algorithm for a restricted class of graph structures that are flexible enough to cover almost all treebank structures, yet are restricted enough to admit efficient learning and inference.
In particular, we consider directed, acyclic, one-endpoint-crossing graph structures, which cover most dislocation, shared argumentation, and similar tree-violating linguistic phenomena.
We describe how to convert phrase structure parses, including traces, to our new representation, in a reversible manner.
Our dynamic program uniquely decomposes structures, is sound and complete with respect to a subset of the class of one-endpoint-crossing graphs, and covers $97.7\%$ of the Penn English treebank.
We also implement a proof-of-concept parser that recovers a range of null elements and trace types.

\subsection{Old intro}

Many syntactic representations use graphs and/or discontinuous structures, such as traces in Government and Binding theory, and f-structure in Lexical Functional Grammar \cite{gb,Bresnan:1982}.
Sentences in the Penn Treebank \cite{ptb} have a core projective tree structure, and trace edges that represent control structures, wh-movement and more.
However, most parsers and the standard evaluation metric ignore these edges and all null elements.
By leaving out parts of the structure, they fail to provide all predicates to downstream tasks such as question answering.
While there has been work on capturing some parts of this extra structure, it has generally either been through post-processing on trees
\cite{Johnson:2002,Jijkoun:2003,Campbell:2004,Levy:2004,Gabbard:2006},
or has only captured a limited set of phenomena via grammar augmentation
\cite{collins:1997,dienes-dubey:2003,schmid:2006,cai-chiang-goldberg:2011}.
%%%In both cases phenomena such as shared argumentation are completely ignored.
%%%Similarly, most work on the Abstract Meaning Representation \cite{amr}, has removed edges to turn all structures into trees.

\begin{figure}
  \centering
  \scalebox{0.65}{
  \begin{tikzpicture}
[ every node/.style={
    node distance=2ex,
    inner sep=0pt
  },
  structural/.style={
    ->,
    >=Stealth,
    thin,
  },
  trace/.style={
    ->,
    >=Stealth,
    thin,
    dashed
  },
]

% Split head form
  \node (sntR0) at (0, 1.8) {\strut ROOT};

  \node (snt00) [right=1em of sntR0.east] {\strut NP\textsubscript{SBJ}};
  \node (sw0) [below=-0.3ex of snt00] {\strut Sam};
%%%  \draw (sw0.north) -- (snt00.south);

  \node (snt10) [right=1em of snt00.east] {\strut S};
  \node (snt11) [right=1ex of snt10.east] {\strut VP};
  \draw (snt10.east) -- (snt11.west);
  \node (sw1a) at ($(snt10.west)!0.5!(snt11.east)$) {\strut};
  \node (sw1) [below=-0.3ex of sw1a] {\strut told};
%%%  \draw (sw1.north) -- (sw1a.south);

  \node (snt20) [right=1em of snt11.east] {\strut NP};
  \node (sw2) [below=-0.3ex of snt20] {\strut Alex};
%%%  \draw (sw2.north) -- (snt20.south);

  \node (snt30) [right=1.5em of snt20.east] {\strut -};
  \node (sw3) [below=-0.3ex of snt30] {\strut to};
%%%  \draw (sw3.north) -- (snt30.south);

  \node (snt50) [right=1em of snt30.east] {\strut S};
  \node (snt51) [right=1ex of snt50.east] {\strut VP};
  \draw (snt50.east) -- (snt51.west);
  \node (snt52) [right=1ex of snt51.east] {\strut VP};
  \draw (snt51.east) -- (snt52.west);
  \node (sw5a) at ($(snt50.west)!0.5!(snt52.east)$) {\strut};
  \node (sw5) [below=-0.3ex of sw5a] {\strut run};
%%%  \draw (sw5.north) -- (sw5a.south);

  \draw [structural,out=60,in=120] (snt00.north) to (snt10.north west);
  \draw [structural,out=120,in=60] (snt10.north) to (sntR0.north);
  \draw [structural,out=120,in=60] (snt20.north) to (snt11.north east);
  \draw [trace,out=30,in=150] (snt20.north) to node[midway,above=1pt] {\scriptsize NP\textsubscript{SBJ}} (snt50.north west);
  \draw [structural,out=120,in=60] (snt50.north) to (snt11.north);
  \draw [structural,out=60,in=120] (snt30.north) to (snt51.north);

% Standard form
  \node (w0) [above=8ex of sw0] {\strut Sam};
  \node (w1) [right=3ex of w0.east] {\strut told};
  \node (w2) [right=3ex of w1.east] {\strut Alex};
  \node (w3) [right=3ex of w2.east] {\strut};
  \node (w4) [right=3ex of w3.east] {\strut to};
  \node (w5) [right=3ex of w4.east] {\strut run};

  \node (nt00) [above=of w0.north] {\strut};
  \node (nt10) [above=of w1.north] {\strut};
  \node (nt20) [above=of w2.north] {\strut};
  \node (nt30) [above=of w3.north] {\strut *\textsubscript{1}};
  \node (nt40) [above=of w4.north] {\strut};
  \node (nt50) [above=of w5.north] {\strut VP};

  \node (nt01) [above=of nt00.north] {\strut};
  \node (nt11) [above=of nt10.north] {\strut};
  \node (nt21) [above=of nt20.north] {\strut};
  \node (nt31) [above=of nt30.north] {\strut NP\textsubscript{SBJ}};
  \path (nt40.north) -- node[above=2ex] (nt41-51) {\strut VP} (nt50.north);

  \node (nt02) [above=of nt01.north] {\strut};
  \node (nt12) [above=of nt11.north] {\strut};
  \node (nt22) [above=of nt21.north] {\strut NP\textsubscript{1}};
  \path (nt31.north) -- node[above=2ex] (nt32-52) {\strut S} (nt41-51.north);

  \node (nt03) [above=of nt02.north] {\strut NP\textsubscript{SBJ}};
  \path (nt12.north) -- node[above=2ex] (nt13-53) {\strut VP} (nt32-52.north);

  \path (nt03.north) -- node[above=2ex] (nt04-54) {\strut S} (nt13-53.north);

  \draw (w0.north) -- (nt03.south);
  \draw (w1.north) -- (nt12.north) -- (nt13-53.south) -- (nt32-52.north);
  \draw (w2.north) -- (nt22.south);
  \draw (w4.north) -- (nt40.north) -- (nt41-51.south) -- (nt50.north);
  \draw (w5.north) -- (nt50.south);

  \draw (nt03.north) -- (nt04-54.south) -- (nt13-53.north);
  \draw (nt22.north) -- (nt13-53.south);
  \draw (nt30.north) -- (nt31.south);
  \draw (nt31.north) -- (nt32-52.south) -- (nt41-51.north);

  \node (figB) [left=4ex of sntR0] {\strut \large (b)};
  \node (figA) at (figB |- nt31) {\strut \large (a)};
\end{tikzpicture}

  }
  \vspace{-5mm}
  \caption{\label{fig:repr}
    Parse representation: (a) constituency (b) ours.
  }
\end{figure}

We propose a new parse representation and a new algorithm that can efficiently consider almost all syntactic phenomena.
Our representation is an extension of TAG-based tree representations \cite{cck,Shen:2007}, modified to represent graphs and designed to maximize coverage under a new class of graphs.
Our algorithm extends a non-projective tree parsing algorithm \cite{ec} to graph structures, with improvements to avoid derivational ambiguity.

Our representation, shown in Figure~\ref{fig:repr}b, consists of complex tags composed of non-terminals, and edges indicating attachment.
In this form, traces can create problematic structures such as directed cycles, but we show how careful choice of head rules can minimize such issues.

Our algorithm runs in time $O(n^4)$ under a first-order model.
We also introduce extensions that ensure parses contain a directed projective tree of non-trace edges.
We implemented a proof-of-concept parser with a basic first-order model, scoring $88.3$ on trees in section 22, and recovering a range of trace types.

Together, our representation and algorithm form an inference method that can cover $97.7\%$ of sentences, far above the coverage of projective tree algorithms ($46.8\%$).
