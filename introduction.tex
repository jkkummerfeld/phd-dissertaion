\chapter{Introduction}

Communication is a fundamental part of human society, enabling the transfer of information between people.
Textual language in particular is found throughout our daily lives;
when asking a question, we type it into a search engine and usually read an answer from either Wikipedia or discussions between other people;
to find out what is happening in the world we read newspapers or listen to a newsreader;
for entertainment we read stories, listen to podcasts, or watch TV shows (virtually all of which contain dialogue);
and for personal communcation we write electronic messages in a variety of formats.
One of the keys to all of this communication is the use of structure to convey meaning.
Part of this structure is visible--by convention you are looking at this page from left to right and top to bottom, treating each connected component of ink (or light) as a character, combining a sequence of characters into words, and so on to progressively larger levels of structure.
The focus of this dissertation, syntax, is the mostly invisible structure that exists somewhere between characters and sentences.
People are able to identify syntactic structure by drawing on knowledge from a range of sources, and in the process they discard a vast array of alternative interpretations for a given utterance.
In the past century, artificial symbolic processing by machines has advanced dramatically, to the point where computers can start to engage in textual communication, both understanding what people have written and writing back.
Approaching human-level communication will require systems to resolve syntactic ambiguities in text, either explicitly with an interpretable syntactic structure or implicitly with some other intermediate representation.

Over the past two decades there has been rapid development in systems for the syntactic parsing task, where the input is a single sentence and the output is a structure that encodes syntactic relationships between words in the sentence.
This development has largely been driven by development of new statistical methods for constructing models of language from resources manually labeled with syntactic structures by linguists.
In addition to variations in statistical and engineering approaches, research has explored various syntactic representations.
These representations are based on different linguistic theories, which sometimes encode the same phenomena in different ways and sometimes differ on what phenomena should be disambiguated within syntax.

In this thesis, we present new algorithms that extend the capabilities of various systems related to syntax.
The first area of research we consider is how we evaluate our systems to understand their strengths and weaknesses.
We propose a search algorithm that finds the set of errors present in a parse.
Using the algorithm, we compare a range of systems, a range of text domains, and two languages.
The second aspect of syntax we consider is conversion between different linguistic formalisms.
Our system uses a bottom-up approach that more flexible and more effective than previous work.
Finally, we explore how to extend parsing algorithms beyond tree structured output.
We introduce a new algorithm that can efficiently find the max or sum over all possible structures for a sentence, scored with an edge-factored model.
We define a new representation that is compatible with the algorithm while representing virtually all forms of linguistic structure in the Penn Treebank.

\section{Syntax}
\label{sec:syntax}

\begin{figure}
  TODO FIGURE
  \caption{
  \label{fig:syntax}
  dependency, 
%%%  Ellen nsubj (x2), one dashed
%%%  enjoys 
%%%  reading xcomp
  constituency, 
%%%  (ROOT
%%%    (S
%%%      (NP-1 Ellen)
%%%      (VP enjoys
%%%          (S
%%%            (NP *-1)
%%%            (VP running)))))
  and categorial forms of the sentence \textexample{Ellen enjoys running}.
%%%  Ellen NP
%%%  enjoys (S\NP)/(S\NP)
%%%  reading S\NP
%%% beneath
%%%  Ellen NP -> S / (S\NP)
%%%  enjoys (S\NP)/(S\NP)
%%%  reading S\NP 
  }
\end{figure}

One of the core research areas in linguistics is syntax, the study of processes that determine word order in sentences.
For the purposes of this thesis, it is important to understand a few general properties of the syntactic theories we consider.
Each theory encodes relationships between words using structure, but the form of those structures varies considerably.
Figure~\ref{fig:syntax} shows how the sentence \textexample{Ellen enjoys running} is represented in the three syntactic formalisms we use.

In the leftmost case, which uses dependency grammar \parencite[\depgr][]{dependency-grammar}, each edge indicates a relationship between two words.
The label expresses the type of dependency, out of 40-50 types \parencite{ud,sd}.
The arrow points from the word that is the \textit{dependent} / \textit{child} to the word that is the \textit{head} / \textit{parent}\footnote{The direction of the arrow is a convention we are following from linguistics. Note that this is the reverse of the convention in graph theory.}.
The dashed edge is often excluded in order to make the edges form a tree (a structure that connects all words, but does not contain a cycle).

The middle case, which uses Government and Binding Theory \parencite[\gb][]{gb}, has a hierarchical phrase structure.
Each symbol is a \textit{constituent}, capturing the idea that the set of words beneath it constitute a single functional unit.
Consistituents are linked together to form larger constituents according to rules.
This figure also shows a null element between \textexample{enjoys} and \textexample{running}, which is used to encode the relationship between \textexample{Ellen} and \textexample{running}.
As with the dashed edge in the dependency case, the null element is often removed from this structure to make it a tree.

The rightmost case, categorial grammar \parencite{categorial-grammar}, also has a hierarchical structure, but uses complex lexical categories that are then combined according to a small set of inference rules.
This particular example uses Combinatory Categorial Grammar \parencite[\ccg][]{Steedman:2000}, a variant in which combinatory logic is used to construct both syntactic and semantic forms in parallel.
Immediately beneath each word is its lexical category, which can either by an atomic symbol, or a complex structured combination of symbols.
A small set of generic combinators define how pairs of adjacent categories can be combined to produce a new structure.
The two structures shown have different surface structures, but produce the same semantic expression.
This ability to have different structures encode the same meaning is a form of derivational or spurious ambiguity\footnote{\textcite{Steedman:2000} explores the possibility that this ambiguity could encode variations in prosody.}.
Note that unlike the previous two approaches, here the connection between \textexample{Ellen} and \textexample{running} is retained while keeping the structure a tree.
This is possible because the relation has been threaded through the tree via the categories.

%%%Comparing these three representations of syntax we can see some axes of variation:
%%%\begin{itemize}
%%%  \item[Lexical heads]
%%%  Explicit in \depgr. Implicitly encoded by \ccg via the direction of composition. Not represented in \gb.
%%%  \item[Constituents]
%%%  Explicit in \gb and \ccg. Can be inferred in \depgr by considering all descendants of a word (note, not necessarily a continuous span).
%%%  \item[Number of rules vs. number of labels]
%%%  \ccg has a small set of general combinators, but many categories. \gb has many rules, but only a few labels. \depgr has a small number of both.
%%%  \item[Structural properties]
%%%  \ccg derivations are always trees, but have spurious ambiguity, while \depgr and \gb do not have spurious ambiguity but cannot represent all relations in a tree.
%%%\end{itemize}

While syntactic formalisms are used in a variety of ways, our focus is on the automatic production of syntactic structures by computer programs.
These programs take a sentence as input, consider possible structures and return the one that is best according to some model\footnote{The sentence is assumed to be grammatical--a parse is always returned.}.
Individually considering every possible structure for a sentence is infeasible.
Instead, programs incrementally build the parse, and as a result they either lose the guarantee of finding the optimal parse according to the model, or give up flexibility in the definition of the model.
This work follows the second approach, with models being restricted to give the complete structure a score that is the sum of individual scores for each of its components.
With this restriction we can apply dynamic programming methods, where the larger problem (find the optimal parse) is decomposed into independent sub-problems (find the optimal parse for part of the sentence) whose solutions can be used to solve the original problem (take the best solution from one half of the sentence and combine it with the best solution from the other half).
The specific form of dynamic programming applied to the task of finding optimal parses structures is the CKY algorithm (CITE).

\section{Error Analysis}

The standard resource for parsing research is the Wall Street Journal section of the Penn Treebank \parencite{ptb}, a collection of one million words of text from 1989 issues of the Wall Street Journal that have been annotated by experts with syntactic structure in a \gb style.
The standard measure of constituent parser performance is the F-Score, the harmonic mean of precision and recall on labeled nodes in the parse.
Performance on \wsj section 23 has exceeded $90$ F\textsubscript{1} \parencite{Petrov-Klein:2007}, and $92$ F\textsubscript{1} when using self-training and reranking \parencite{McClosky-Charniak-Johnson:2006,Charniak-Johnson:2005}. 
While these results give a useful measure of overall performance, they provide no information about the nature, or relative importance, of the remaining errors.

Broad investigations of parser errors beyond the \parseval metric \parencite{Black-etal:1991} have either focused on specific parsers, \myeg \parencite{Collins:2003}, or have involved conversion to \depgr \parencite{Carroll-etal:1998,King:2003}.
In all of these cases, the analysis has not taken into consideration how a set of errors can have a common cause, \myeg a single mis-attachment can create multiple node errors.

In the first part of the thesis, we propose a new method of error classification using tree transformations.
Errors in the parse tree are repaired using subtree movement, node creation, and node deletion.
Each step in the process is then associated with a linguistically meaningful error type, based on factors such as the node that is moved, its siblings, and parents.  
Using our method we analyse the output of thirteen constituency parsers on newswire.
Some of the frequent error types that we identify are widely recognised as challenging, such as prepositional phrase (PP) attachment.
However, other significant types have not received as much attention, such as clause attachment and modifier attachment.
We also investigate where reranking and self-training improve parsing, and where performance decreases when parsing out-of-domain text.
Previously, these were all analysed only in terms of their impact on F-score.

\section{Fromalism Conversion}

As shown above, there are many ways of expressing syntactic structure.
Extensive work has gone into converting the Penn Treebank \parencite[\ptb][]{Marcus-Marcinkiewicz-Santorini:1993} to other formalisms, such as \hpsg \parencite{Miyao-Ninomiya-Tsujii:2004}, \lfg \parencite{Cahill-etal:2008}, \ltag \parencite{Xia:1999}, and \ccg \parencite{CCGBank},
These conversions are complex processes that render linguistic phenomena in formalism-specific ways.
Tools for reversing these conversions are desirable for downstream parser use and parser comparison.
However, reversing conversions is difficult, as corpus conversions may lose information or smooth over PTB inconsistencies.
Clark and Curran (2009) developed a CCG to PTB conversion that treats the CCG derivation as a phrase structure tree and applies hand-crafted rules to every pair of categories that combine in the derivation.
Because their approach does not exploit the generalisations inherent in the CCG formalism, they must resort to ad-hoc rules over non-local features of the CCG constituents being combined (when a fixed pair of CCG categories correspond to multiple PTB structures).
Even with such rules, they correctly convert only 39.6\% of gold CCGbank derivations.

In the second chapter, we describe a conversion method that assigns a set of bracket instructions to each word based on its CCG category, then follows the CCG derivation, applying and combining instructions at each combinatory step to build a phrase structure tree.
This requires specific instructions for each category (not all pairs), and generic operations for each combinator.
We cover all categories in the development set and correctly convert 51.1\% of sentences.
Unlike Clark and Curran's approach, we require no rules that consider non-local features of constituents, which enables the possibility of simple integration with a CKY-based parser.
The most common errors our approach makes involve nodes for clauses and rare spans such as QPs, NXs, and NACs.
Many of these errors are inconsistencies in the original PTB annotations that are not recoverable.
These issues make evaluating parser output difficult, but our method does enable an improved comparison of \ccg and \ptb parsers.

\section{Graph Parsing}

In Section~\ref{sec:syntax} we saw that some edges in the dependency parse, and the traces in the Government and Binding structure, break context free assumptions, leading to graphs and/or discontinuous structures.
In the Penn Treebank, the traces are distinguished from the core projective tree structure, and are used to represent control structures, wh-movement and more.
However, most parsers and the standard evaluation metric ignore these edges and all null elements.
By leaving out parts of the structure, they fail to provide all predicates to downstream tasks such as question answering.
While there has been work on capturing some parts of this extra structure, it has generally either been through post-processing on trees \parencite{Johnson:2002,Jijkoun:2003,Campbell:2004,Levy:2004,Gabbard:2006}, or has only captured a limited set of phenomena via grammar augmentation \parencite{collins:1997,dienes-dubey:2003,schmid:2006,cai-chiang-goldberg:2011}.
In both cases phenomena such as shared argumentation are completely ignored.
Similarly, most work on the Abstract Meaning Representation \parencite{amr}, has removed edges to turn all structures into trees.

\begin{figure}
  \centering
  \scalebox{1.0}{
    \begin{tikzpicture}
[ every node/.style={
    node distance=2ex,
    inner sep=0pt
  },
  structural/.style={
    ->,
    >=stealth',
    thin,
  },
  trace/.style={
    ->,
    >=stealth',
    thin,
    dashed
  },
]

% Split head form
  \node (sntR0) at (0, 1.8) {\strut ROOT};

  \node (snt00) [right=1em of sntR0.east] {\strut NP\textsubscript{SBJ}};
  \node (sw0) [below=-0.3ex of snt00] {\strut Sam};
%%%  \draw (sw0.north) -- (snt00.south);

  \node (snt10) [right=1em of snt00.east] {\strut S};
  \node (snt11) [right=1ex of snt10.east] {\strut VP};
  \draw (snt10.east) -- (snt11.west);
  \node (sw1a) at ($(snt10.west)!0.5!(snt11.east)$) {\strut};
  \node (sw1) [below=-0.3ex of sw1a] {\strut told};
%%%  \draw (sw1.north) -- (sw1a.south);

  \node (snt20) [right=1em of snt11.east] {\strut NP};
  \node (sw2) [below=-0.3ex of snt20] {\strut Alex};
%%%  \draw (sw2.north) -- (snt20.south);

  \node (snt30) [right=1.5em of snt20.east] {\strut -};
  \node (sw3) [below=-0.3ex of snt30] {\strut to};
%%%  \draw (sw3.north) -- (snt30.south);

  \node (snt50) [right=1em of snt30.east] {\strut S};
  \node (snt51) [right=1ex of snt50.east] {\strut VP};
  \draw (snt50.east) -- (snt51.west);
  \node (snt52) [right=1ex of snt51.east] {\strut VP};
  \draw (snt51.east) -- (snt52.west);
  \node (sw5a) at ($(snt50.west)!0.5!(snt52.east)$) {\strut};
  \node (sw5) [below=-0.3ex of sw5a] {\strut run};
%%%  \draw (sw5.north) -- (sw5a.south);

  \draw [structural,out=60,in=120] (snt00.north) to (snt10.north west);
  \draw [structural,out=120,in=60] (snt10.north) to (sntR0.north);
  \draw [structural,out=120,in=60] (snt20.north) to (snt11.north east);
  \draw [trace,out=30,in=150] (snt20.north) to node[midway,above=1pt] {\scriptsize NP\textsubscript{SBJ}} (snt50.north west);
  \draw [structural,out=120,in=60] (snt50.north) to (snt11.north);
  \draw [structural,out=60,in=120] (snt30.north) to (snt51.north);

% Standard form
  \node (w0) [above=8ex of sw0] {\strut Sam};
  \node (w1) [right=3ex of w0.east] {\strut told};
  \node (w2) [right=3ex of w1.east] {\strut Alex};
  \node (w3) [right=3ex of w2.east] {\strut};
  \node (w4) [right=3ex of w3.east] {\strut to};
  \node (w5) [right=3ex of w4.east] {\strut run};

  \node (nt00) [above=of w0.north] {\strut};
  \node (nt10) [above=of w1.north] {\strut};
  \node (nt20) [above=of w2.north] {\strut};
  \node (nt30) [above=of w3.north] {\strut *\textsubscript{1}};
  \node (nt40) [above=of w4.north] {\strut};
  \node (nt50) [above=of w5.north] {\strut VP};

  \node (nt01) [above=of nt00.north] {\strut};
  \node (nt11) [above=of nt10.north] {\strut};
  \node (nt21) [above=of nt20.north] {\strut};
  \node (nt31) [above=of nt30.north] {\strut NP\textsubscript{SBJ}};
  \path (nt40.north) -- node[above=2ex] (nt41-51) {\strut VP} (nt50.north);

  \node (nt02) [above=of nt01.north] {\strut};
  \node (nt12) [above=of nt11.north] {\strut};
  \node (nt22) [above=of nt21.north] {\strut NP\textsubscript{1}};
  \path (nt31.north) -- node[above=2ex] (nt32-52) {\strut S} (nt41-51.north);

  \node (nt03) [above=of nt02.north] {\strut NP\textsubscript{SBJ}};
  \path (nt12.north) -- node[above=2ex] (nt13-53) {\strut VP} (nt32-52.north);

  \path (nt03.north) -- node[above=2ex] (nt04-54) {\strut S} (nt13-53.north);

  \draw (w0.north) -- (nt03.south);
  \draw (w1.north) -- (nt12.north) -- (nt13-53.south) -- (nt32-52.north);
  \draw (w2.north) -- (nt22.south);
  \draw (w4.north) -- (nt40.north) -- (nt41-51.south) -- (nt50.north);
  \draw (w5.north) -- (nt50.south);

  \draw (nt03.north) -- (nt04-54.south) -- (nt13-53.north);
  \draw (nt22.north) -- (nt13-53.south);
  \draw (nt30.north) -- (nt31.south);
  \draw (nt31.north) -- (nt32-52.south) -- (nt41-51.north);

  \node (figB) [left=4ex of sntR0] {\strut \large (b)};
  \node (figA) at (figB |- nt31) {\strut \large (a)};
\end{tikzpicture}

  }
  \caption{\label{fig:repr}
    Parse representation: (a) constituency (b) ours.
  }
\end{figure}

In the fianl chapter, we propose a new parse representation and a new algorithm that can efficiently consider almost all observed syntactic phenomena.
Our representation is an extension of TAG-based tree representations \parencite{cck,Shen:2007}, modified to represent graphs and designed to maximize coverage under a new class of graphs.
Our algorithm extends a non-projective tree parsing algorithm \parencite{ec} to graph structures, with improvements to avoid derivational ambiguity.

Our representation, shown in Figure~\ref{fig:repr}b, consists of complex tags composed of non-terminals, and edges indicating attachment.
In this form, traces can create problematic structures such as directed cycles, but we show how careful choice of head rules can minimize such issues.

Our algorithm runs in time $O(n^4)$ under a first-order model.
We also introduce extensions that ensure parses contain a directed projective tree of non-trace edges.
We implemented a proof-of-concept parser with a basic first-order model, scoring $88.3$ on trees in section 22, and recovering a range of trace types.
Together, our representation and algorithm form an inference method that can cover $97.7\%$ of sentences, far above the coverage of projective tree algorithms ($46.8\%$).

\section{Contributions of This Dissertation}

Our contributions are a set of novel algorithms and experimental results and analysis using those algorithms.
First, we define a new algorithm for error analysis of constituency parsing output.
We implemtn the algorithm and use it to perform an analysis of current parsing effectiveness, considering a wide range of systems, multiple domains, and two languages.
Second, we present a new algorithm for transforming \ccg derivations into \gb parses.
Our approach is significantly more accurate than previous work and has desireable algorithmic properties.
Finally, we describe the first algorithm for inference over the space of \gb graph structures.
We show how to efficiently implement the algorithm, and discuss results and remaining challenges.

