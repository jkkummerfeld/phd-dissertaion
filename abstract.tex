\begin{abstract}

Representation of syntactic structure is a core area of research in Computational Linguistics, disambiguating distinctions in meaning that are crucial for correct interpretation of language.
Development of algorithms and statistical models over the past two decades has led to systems that are accurate enough to be deployed in industry, playing a key role in products such as Google Search and Apple Siri, and part of the foundation of startups such as SpaCy and WriteLab.
However, syntactic parsers today are usually constrained to tree representations of language, and performance is interpreted through a single metric that coneys no linguistic information regarding remaining errors.

In this dissertation, we present new tools for manipulating syntactic structures: modifying existing structures and making new ones.
Each tool is based on a core algorithm and adaptations specific to linguistic structure.

First, we consider a search algorithm that, given two structures, will find a sequence of modifications leading from one structure to the other.
We apply this algorithm to syntactic error analysis, where one structure is the output of a parser, the other is the correct parse, and each modification corresponds to fixing one error.
We constructed a tool based on the algorithm and analysed variations in behaviour between parsers, types of text, and languages.
Within the CL community there are a range of widely-held beliefs regarding the prevalence of certain errors and hence the nature of unsolved challenges in parsing.
Our analysis provided new empirical evidence, supporting some beliefs, such as the high frequency of prepositional phrase attachment errors, while reframing others, for example coordination scope errors do not have a major impact on performance though resolving them remains a challenging problem.

Next, we describe a mapping algorithm that converts one syntactic representation into another.
Specifically, we map from Combinatory Categorial Grammar derivations to Phrase Structure trees.
Our approach follows the philosophy of \ccg, defining a phrase structure for each lexical category and general rules for combining phrase structures.
This produces significantly more accurate phrase structures than past work, and has the advantage that the mapping decomposes in the same way as the CKY parsing algorithm.

Finally, we present a dynamic programming algorithm, which constructs the graph structure that has the highest score under an edge-factored scoring function.
In the process of defining a parse representation compatible with the algorithm, we found that certain linguistic distinctions could dramatically impact coverage.
We also show various ways to modify the algorithm to improve performance by exploiting properties of observed linguistic structure.
This approach to syntactic parsing is the first to cover virtually all structure encoded in the Penn Treebank.


\end{abstract}
