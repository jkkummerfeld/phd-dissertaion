\begin{acknowledgements}

Every doctorate is a long journey, developing new understanding of the universe and yourself.
I've had an incredible group of people helping and supporting me on my path.

First, my adviser, Dan Klein.
%%%, whose guidance covered everything from low-level data structure implementation details, through general research directions, to navigating the challenges of being a course instructor.
Completing a doctorate is about learning many new skills, and Dan taught me all of them, either explicitly through guidance or implicitly through the example he set of how to do great research, present your work, run a group, teach, and more.

Next, I'd like to recognise the time and effort my committee put into their extremely thoughtful and detailed comments on this thesis.
As a result, every chapter explains the work more clearly and fully, and more effectively draws connections with other research.

I have also been fortunate to work with fantastic people.
As well as my adviser, the research presented here involved three key collaborators: Daniel Tse, David Hall, and James Curran.
Your expertise was crucial, on everything from explaining Chinese syntax to showing how to squeeze as much content into a paper as possible.

Beyond my main collaborators, I have been assisted immensely by conversations with everyone who has been a part of the Berkeley NLP group during my PhD:
Jacob Andreas, Mohit Bansal, Taylor Berg-Kirkpatrick, David Burkett, Greg Durrett, Daniel Fried, Dave Golland, David Hall, Nikita Kitaev, Percy Liang, Dominick Ng, Adam Pauls, Max Rabinovich, and Mitchell Stern.
In particular, I am grateful to Greg, who I met at the Berkeley visit day, spent most of my degree working (literally) alongside, and have immense respect for as both a researcher and a friend.

Another form of support was provided by the General Sir John Monash Foundation, which connected me to an incredible group of Australians pursuing exciting careers across every field.
I was also supported by grants with the Berkeley Security group, which gave me valuable experience working on a multi-disciplinary team with fascinating data.

Yet another form of support came from the circle of friends that I was a part of in Berkeley.
In hindsight, I was incredibly lucky when I mailed the rest of the incoming students and asked if anyone wanted to live in a large house together.
Not only did you become some of my closest friends, but our home became the centre of a larger group of friends that had many great adventures together.
I will miss the community we formed, but am sure that our connection will be just as strong every time we are together again.

Looking back further, I want to acknowledge the people who were role models for me when I chose to follow the academic path.
There are too many people to name here, including research advisers on undergraduate projects across computer science, chemistry, and physics, everyone at the JHU workshop, and my friends who started down the doctoral path before me.
One group I would like to specifically mention is my family.
Each of you inspire me in different ways, and all of you have encouraged and supported me my entire life.

Finally, along the way I found more than just interesting ideas and great friends.
I met my now-fianc\'{e}e, Ellen.
You have been supportive, patient, generous, and many more positive adjectives.

Thank you.

\end{acknowledgements}
