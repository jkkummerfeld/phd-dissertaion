\chapter{Conclusion}

This dissertation has explored new algorithms for manipulating syntactic structure.
These have extended our understanding and expanded the capacity of our tools.
First, we looked at improving our understanding of the mistakes parsers make.
Our error analysis method gives a summary of the types of errors in system output, with categories that are easily interpretable.
We applied the technique to a range of systems, writing styles, and languages, confirming previously anecdotal claims such as the prevalence of prepositional phrase attachment errors.
Next, we considered the connections between syntactic representations.
Our method of converting from \ccg into \ptb-style trees improved significantly over prior work.
Finally, we explored parsing with aspects of syntax that are typically ignored.
Our algorithm is the first to incorporate virtually all aspects of structure in the \ptb into a single inference method.

There are a range of possible extensions for each component of this work.
Both the error analysis and conversion systems are designed around trees, and could be extended to cover graph structures.
Doing so would require breaking assumptions built-in to the algorithms, but would be particularly illuminating regarding the challenge of long-distance dependencies.
As for the graph parsing algorithm, there are a several interesting possibilities.
In the structure of this dissertation, the parsing algorithm is kept separate from the definition of the syntactic representation we use.
That choice was intentional, as the algorithm is a general parsing approach that could also be applied to exciting new resources like the Abstract Meaning Representation.
Having non-local structure available also opens new opportunities for downstream applications.
Finally, in the implementation of the algorithm we found that the vast majority of the deduction rules are not needed to cover the structures seen in language.
More tightly characterizing the space of structures remains a fascinating open question.
